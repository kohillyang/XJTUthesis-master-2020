% multiple1902 <multiple1902@gmail.com>
% denotation.tex
% Copyright 2011~2012, multiple1902 (Weisi Dai)
% https://code.google.com/p/xjtuthesis/
% 
% It is strongly recommended that you read documentations located at
%   http://code.google.com/p/xjtuthesis/wiki/Landing?tm=6
% in advance of your compilation if you have not read them before.
%
% This work may be distributed and/or modified under the
% conditions of the LaTeX Project Public License, either version 1.3
% of this license or (at your option) any later version.
% The latest version of this license is in
%   http://www.latex-project.org/lppl.txt
% and version 1.3 or later is part of all distributions of LaTeX
% version 2005/12/01 or later.
%
% This work has the LPPL maintenance status `maintained'.
% 
% The Current Maintainer of this work is Weisi Dai.
%
\begin{denotation}
        \item[符号1]    解释1
        \item[符号2]    解释2
\end{denotation}
如果论文中使用了大量的物理量符号、标志、缩略词、专门计量单位、自定义名词和术语等,应将全文中常用的这些符号及意义列出。如果上述符号和缩略词使用数量不多,可以不设专门的主要符号表,但在论文中出现时须加以说明。
论文中主要符号应全部采用法定单位,特别要严格执行GB3100~3102—93有关“量和单位”的规定。单位名称的书写,可以采用国际通用符号,也可以用中文名称,但全文应统一,不得两种混用。
缩略词应列出中英文全称。
主要符号表正文统一左缩进一个字符。
符号表排序方法:先按拉丁字母大写、小写排序,再按希腊字母大写、小写排序。
部分内容非强制性要求,如果论文中所用符号不多,可以省略《主要符号表》