\chapter{浮动格式}
\echapter{Float Objects}
    金溪民方仲永,世隶耕。仲永生五年,未尝识书具,忽啼求之。父异焉,借旁近与之,即书诗四句,并自为其名。其诗以养父母、收族为意,传一乡秀才观之。自是指物作诗立就,其文理皆有可观者。邑人奇之,稍稍宾客其父,或以钱币乞之。父利其然也,日扳仲永环谒于邑人,不使学。

    余闻之也久。明道中,从先人还家,于舅家见之,十二三矣。令作诗,不能称前时之闻。又七年,还自扬州,复到舅家问焉。曰:“泯然众人矣。”

    王子曰:仲永之通悟,受之天也。其受之天也,贤于才人远矣。卒之为众人,则其受于人者不至也。彼其受之天也,如此其贤也,不受之人,且为众人;今夫不受之天,固众人,又不受之人,得为众人而已耶?

    \section{图片}
    \esection{Images}
        \begin{figure}[!ht]
          \centering
          \includegraphics[width=6.67cm]{XJTU.pdf}
          \caption{西安交通大学}
          \label{fig:xjtu}
        \end{figure}

        \begin{figure}[!ht]
          \begin{minipage}{0.45\textwidth}
              \centering
              \includegraphics[width=6.67cm]{XJTU.pdf}
              \caption{西安交通大学}
              \label{fig:xjtu-left}
          \end{minipage}
          \begin{minipage}{0.45\textwidth}
              \centering
              \includegraphics[width=6.67cm]{XJTU.pdf}
              \caption{西安交通大学}
              \label{fig:xjtu-right}
          \end{minipage}
        \end{figure}
          
        \begin{figure}[!ht]
          \centering
          \subfloat[果毅力行]{
              \includegraphics[width=6.67cm]{XJTU.pdf}
              \label{fig:xjtu-sub-left}}
          \subfloat[忠恕任事]{
              \includegraphics[width=6.67cm]{XJTU.pdf}
              \label{fig:xjtu-sub-right}}
          \caption{子图}
        \end{figure}
          

    \section{表格}
    \esection{Tables}
        \begin{table}[!ht]
          \centering
          \caption{一个简单的表格}
          \label{tab:simple}
          \wuhao
          \begin{tabularx}{\linewidth}{XXXXX} \toprule 
                & 一月 & 二月 & 三月 & 合计 \\ \midrule
           东部 &    7 &    7 &    5 &   19 \\ 
           西部 &    6 &    4 &    7 &   17 \\ 
           南部 &    8 &    7 &    9 &   24 \\ 
       \bf 合计 &   21 &   18 &   21 &   60 \\ \bottomrule
          \end{tabularx}
        \end{table}


        \begin{table}[!ht]
          %\begin{minipage}{\textwidth}
          \begin{threeparttable}[h]
            \centering
            \caption{包含脚注的表格}
            \label{tab:with-footnote}
            \wuhao
            \begin{tabularx}{\linewidth}{XXXXX} \toprule 
                  & 一月 & 二月 & 三月 & 合计 \\ \midrule
                  东部 &    7\tnote{1}
                                &    7 &    5 &   19 \\ 
             西部 &    6 &    4 &    7 &   17 \\ 
             南部 &    8 &    7 &    9 &   24 \\ 
             \bf 合计\tnote{2}
                  &   21 &   18 &   21 &   60 \\ \bottomrule
            \end{tabularx}
          %\end{minipage}
          \begin{tablenotes}
          \item[1] 数据来自Word 97.
          \item[2] Computed by \textsl{Mathematica} 8.
          \end{tablenotes}
          \end{threeparttable}
        \end{table}

        \begin{table}[!ht]
          \centering
          \caption{稍微复杂一点的表格}
          \label{tab:complex}
          \wuhao
          \begin{tabularx}{\linewidth}{XXXXX} \toprule 
                & \multicolumn{3}{c}{这是一句废话} &  \\ \cmidrule{2-4}
                & 一月 & 二月 & 三月 & 合计 \\ \midrule
           东部 &    7 &    7 &    5 &   19 \\ 
           西部 &    6 &    4 &    7 &   17 \\ 
           南部 &    8 &    7 &    9 &   24 \\ 
       \bf 合计 &   21 &   18 &   21 &   60 \\ \bottomrule
          \end{tabularx}
        \end{table}

        我制作了一个简单的表格(表\ref{tab:simple})。


\chapter{公式环境}
\echapter{Formula Environment}

    \begin{axiom}
        \rm 两点间直线段距离最短。  
        \begin{align}
            x&\equiv y+1\pmod{m^2}\\
            x&\equiv y+1\mod{m^2}\\
            x&\equiv y+1\pod{m^2}
        \end{align}
    \end{axiom}

    \begin{remark}
    \rm 对齐的公式示例,它还同时演示了标号。
    \begin{align}
    \begin{split} 
    \varphi(x,z)
    &=z-\gamma_{10}x-\gamma_{mn}x^mz^n\\
    &=z-Mr^{-1}x-Mr^{-(m+n)}x^mz^n
    \end{split} \notag \\
    \noindent\zeta^1&=(\xi^1)^2,\\
    \zeta^1 &=\xi^0\xi^1,\\
    \zeta^2 &=(\xi^1)^2,
    \end{align}
    \end{remark}

    \begin{theorem}
      \rm 对于直角三角形$ABC$, 若$a<c$且$b<c$, 则有
        \begin{equation}
          a^2+b^2=c^2
        \end{equation}
    \end{theorem}


    \begin{exercise}
          \rm 请列出温家宝的所有影视作品。
    \end{exercise}
    
    {
        \color{red} 公式引用应为式(\ref{equ:chap1:bayes})的形式,部分老师可能要求为式~(\ref{equ:chap1:bayes}),请自行确定。
    }
    
    贝叶斯公式如(\ref{equ:chap1:bayes}),其中 $p(y|\mathbf{x})$ 为后验;
    $p(\mathbf{x})$ 为先验;分母 $p(\mathbf{x})$ 为归一化因子。
    \begin{equation}
        \label{equ:chap1:bayes}
        p(y|\mathbf{x}) = \frac{p(\mathbf{x},y)}{p(\mathbf{x})}=
        \frac{p(\mathbf{x}|y)p(y)}{p(\mathbf{x})} 
    \end{equation}