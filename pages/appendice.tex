\setcounter{chapter}{1}
\xjtuspchapter{附录A~公式定理证明}{\vspace{-4mm}附录A\quad 公式定理证明}{AppendProofs of Equations and Theoremsices}
\renewcommand{\thefigure}{\Alph{chapter}-\arabic{figure}}
\renewcommand{\thetable}{\Alph{chapter}-\arabic{table}}
\renewcommand{\theequation}{\Alph{chapter}-\arabic{equation}}
\renewcommand{\thelstlisting}{\Alph{chapter}-\arabic{lstlisting}}
\renewcommand\thealgorithm{\Alph{chapter}-\arabic{algorithm}}
\renewcommand\thesection{\Alph{chapter}.\arabic{section}}
{
    \color{red}
    附录非强制性要求,如果论文中没有附录,可以省略《附录》。
}
% 单个附录用Appendix
% \xjtuspchapter{附录}{附录}{Appendix}
附录编号依次编为附录A,附录B。附录标题各占一行,按一级标题编排。每一个附录一般应另起一页编排,如果有多个较短的附录,也可接排。附录中的图表公式另行编排序号,与正文分开,编号前加“附录A-”字样。

附录编号依次编为附录 A,附录 B。附录标题各按一级标题编排。附录中的图、表、公式另行编排序号,编号前加“附录A-”字样。这部分内容非强制性要求,如果论文中没有附录,可以省略。

排版数学定理等环境时最好给环境添加结束符,以明确定理等内容的起止标志,方便阅读。官方模板未对这些内容进行规范,本模板中定义的结束符采用 $\Diamond$,例子的结束符采用 $\blacklozenge$,定理的结束符采用 $\square$,证明的结束符采用 $\blacksquare$。

\begin{enumerate}
	\item 加法交换律,$\forall~x,y \in X$,$x+y = y+x \in X$;
	\item 加法结合律,$\forall~x,y,z \in X$,$(x+y)+z = x+(y+z)$;
	\item 加法的零元,$\exists~0 \in X$,使得 $\forall~x \in X$,$0+x = x$;
	\item 加法的负元,$\forall~x \in X$,$\exists~-x \in X$,使得 $x+(-x) = x-x = 0$。
	\item 数乘结合律,$\forall~\alpha,\beta \in \mathbb{F}$,$\forall~x \in X$,$(\alpha\beta)x = \alpha(\beta x) \in X$;
	\item 数乘分配律,$\forall~\alpha \in \mathbb{F}$,$\forall~x,y \in X$,$\alpha(x+y) = \alpha x + \alpha y$;
	\item 数乘分配律,$\forall~\alpha,\beta \in \mathbb{F}$,$\forall~x \in X$,$(\alpha+\beta)x = \alpha x + \beta x$;
	\item 数乘的幺元,$\exists~1 \in \mathbb{F}$,使得 $\forall~x \in X$,$1 x = x$,
\end{enumerate}

\xjtuspchapter{附录B~算法与代码}{\vspace{-4mm}附录B\quad 算法与代码}{Algorithms and Codes}
\setcounter{chapter}{2}
对于数学、计算机和电子信息专业,算法和代码也是经常用到的排版技巧,如代码~\ref{list:code_appB_lms}和算法~\ref{algo:lm}所示。
\section{代码}
\esection{Codes}
{\fontsize{10pt}{0.5\baselineskip}\selectfont
	\begin{lstlisting}[caption={空时~LMS~算法~Verilog~模块端口声明},label={list:code_appB_lms}]
	module stap_lms
	#(
	parameter      M                = 4,    // number of antennas
	               L                = 5,    // length of FIR filter
	               W_IN             = 18,   // wordlength of input data
	               W_OUT            = 18,   // wordlength of output data
	               W_COEF           = 20    // wordlength of weights
	)(
	output  signed [W_OUT-1:0]      y_i,    // in-phase component of STAP output
	output  signed [W_OUT-1:0]      y_q,    // quadrature component of STAP output
	output                          vout,   // data valid flag of output (high)
	input          [M*W_IN-1:0]     u_i,    // in-phase component of M antennas
	input          [M*W_IN-1:0]     u_q,    // quadrature component of M antennas
	input                           vin,    // data valid flag for input (high)
	);
	\end{lstlisting}
}
\section{算法}
\esection{Algorithms}
\begin{algorithm}[!ht]  
    \caption{ LM算法}
    \label{algo:lm}
    \begin{algorithmic}[1]  
      \While {not found and $k<k_{max}$};
        \If{$\left\|\mathbf{h}_{\mathrm{Im}}\right\| \leq \varepsilon_{2}\left[\|\mathbf{x}\|+\varepsilon_{2}\right]$}
            \State found:=true
        \Else
            \If {$\varrho > 0$}
                \State $\mathbf{x}:=\mathbf{x}_{new}$
            \Else
                \State $\mu:=\mu+\left[F(\mathbf{x})-F\left[\mathbf{x}_{\text {new }}\right]\right] /(2 \alpha)$;
            \EndIf
        \EndIf
      \EndWhile;
    \end{algorithmic}  
\end{algorithm}